\documentclass[14pt]{extarticle}
\usepackage[utf8]{inputenc}
\usepackage[T1]{fontenc}
\usepackage[polish]{babel}
\usepackage{tabularray}
\usepackage{graphicx} % Required for inserting images
\usepackage{tikz-timing}

\setlength{\parindent}{0pt}
\usetikztiminglibrary[new={char=Q,reset char=R}]{counters}

\title{\huge{\textbf{Tura}} \\ \large{Fizyczna implementacja maszyny Turinga}}
\author{Paweł Szymański}
\date{Luty 2026}

\begin{document}
	
	\maketitle
	\tableofcontents
	\newpage
	
	\section{Wstęp}
	Niewątpliwie jednym z najważniejszych osiągnięć ludzkości XX wieku jest wynalazek będący abstrakcyjnym modelem matematycznym. Mowa oczywiście o maszynie Turinga. Dzięki temu rewolucyjnemu wynalazkowi dziś jako ludzkość jesteśmy w stanie automatyzować zaawansowane zagadnienia, jak na przykład obróbka i montaż produktów w fabrykach czy rozwiązywanie skomplikowanych obliczeń matematycznych. Bez maszyn Turinga nie mielibyśmy dziś komputerów, urządzeń tak uniwersalnych i wszechstronnych jak te które znamy, z których korzystamy i z którego najprawdopodobniej czytelnik czyta tę dokumentację. W ramach urzeczywistnienia matematycznego modelu maszyny Turinga powstał projekt ,,Tura'', którego zadaniem jest fizyczna implementacja pewnego modelu maszyny. Z racji, iż podstawowy model zakłada nieograniczoną długość taśmy, implementajca maszyny jako takiej nie jest możliwa. Jednak pewien podzbiór maszyn zakłada skończoną długość taśmy. Tym modelem jest maszyna Turinga ograniczona liniowo. Implementowany model, w celu uproszczenia elektronicznych układów, dodatkowo posiada własność akceptującego stanu zatrzymującego.
	
	\section{Opis zastosowanego modelu}
	Definicja maszyny Turinga $M$ jest następująca:
	
	\[
	M = (Q, \Gamma, \Sigma, \delta, q_0, B, F),
	\]
	gdzie:
	\begin{itemize}
		\item $Q$ - zbiór stanów,
		\item $\Gamma$ - alfabet taśmy,
		\item $\Sigma$ - alfabet słowa wejściowego, $\Sigma \subseteq \Gamma$,
		\item $\delta$ - funkcja przejścia, $\delta: Q \times \Gamma \rightarrow Q \times \Gamma \times D$, gdzie $D$ jest zbiorem możliwych ruchów głowicy maszyny, $D = \{L, R\},$
		\item $q_0$ - stan początkowy, $q_0 \in Q$,
		\item $B$ - symbol pusty, $B \in \Gamma \backslash \Sigma$,
		\item $F$ - zbiór stanów akceptujących.
	\end{itemize}
	Żeby znaleźć jakieś intuicyjne wyobrażenie jak taka maszyna może wyglądać i funkcjonować możemy wyobrazić sobie skrzynkę, w której zamknięty jest układ sterujący urządzeniem oraz przede wszystkim przechowujący jego obecny stan. Obok skrzynki umieszczona jest taśma z symbolami. Ma ona swój początek z jednej strony, jednak z drugiej nie ma końca - jest jednostronnie nieskończona. Taśmę i skrzynkę łączy głowica - czyta ona lub nadpisuje jeden symbol na taśmie oraz przesuwa się po niej. Na początku działania maszyny głowica znajduje się nad skrajnym symbolem taśmy.
	\newline \newline
	Działanie takiej maszyny można porównać do następującej sekwencji czynów:
	\begin{enumerate}
		\item Głowica odczytuje symbol $s \in \Gamma$ z taśmy.
		\item Skrzynka na podstawie obecnego stanu $q \in Q$ oraz $s$ oblicza nowy stan, symol i kierunek przesunięcia głowicy za pomocą funkcji przejścia:  $\delta(q,s) = (q', s', d)$.
		\item Skrzynka zmienia obecny stan z $q$ na $q'$.
		\item Głowica nadpisuje symbol na taśmie $s$ na $s'$, po czym przesuwa się o jeden symbol się zgodnie z kierunkiem $d$.
		\item Sekwencja ta jest powtarzana dopóty, dopóki maszyna nie osiągnie stanu akceptującego.
	\end{enumerate}
	
	[obrazek]
	
	Fizyczne urządzenie cechuje się dodatkowo ograniczoną długością taśmy oraz jednym stanem akceptującym zatrzymującym działanie maszyny. Formalnie ograniczoną długość taśmy można uzyskać poprzez wstawienie wartowników na jej początku oraz końcu, jednak te działanie pozostawione jest dla użytkownika maszyny do samodzielnej implementacji. Zaś stan akceptujący zatrzymujący działanie maszyny oznacza, że $F$ ma dokładnie jeden stan akceptujący $q_A$, po którego osiągnięciu urządzenie zatrzymuje pracę. 
	
	\section{Model Działania}
	\subsection{Opis Ogólny}
	Działanie maszyny w skrócie opiera się o zaprogramowanie urządzenia, wykonanie obliczeń i pobranie wyniku.
	\begin{itemize}
		\item Programowanie odbywa się poprzez zapisanie wejściowej zawartości taśmy oraz funkcji przejścia w trzech blokach pamięci (bloku symboli taśmy, bloku funkcji przejścia symbolu tasmy oraz bloku funkcji przejścia stanu). 
		\item Wykonanie obliczeń to uruchomienie urządzenia i pozostawienie w działaniu do czasu osiągnięcia stanu akceptującego.
		\item Pobranie wyniku jest odczytem zawartości taśmy po zakończeniu działania maszyny.
	\end{itemize}
	
	\subsection{Opis Szczegółowy}
	Opis szczegółowy wchodzi już w fizyczną implementację urządzenia. Jego celem jest staranny opis modułów, sygnałów, ich wzajemnej interakcji oraz dokładne wytłumaczenie działania maszyny. Opis w znacznej mierze opiera się o schematy KiCad i korzysta ze stosowanych w nich oznaczeń oraz nazw.
	
	\subsubsection{Moduły}
	Maszyna została podzielona na 11 modułów w celu oznakowania jednostek odpowiedzialnych za poszczególne funkcje. Moduły w schematach KiCad zostały otoczone ciemnożłółtymi ramkami i podpisane niebieskim tekstem lub jedynie podpisane niebieskim tekstem w przypadku modułów jednoelementowych.
	\begin{description}
		\item[\texttt{TAPE}] -- Blok pamięci symulujący taśmę. Każda komórka pamięci reprezentuje jeden symbol taśmy. Pojedynczy symbol taśmy składa się z 7 bitów;
		\item[\texttt{TAPE SYMBOL REGISTER}] -- Rejestr przechowujący jeden symbol taśmy. Jest on potrzebny do chwilowego przechowania odczytanego symbolu na czas zapisu nowego do \texttt{TAPE};
		\item[\texttt{SYMBOL TRANSLATOR}] -- Blok pamięci symulujący funkcję przejścia dla zadanego symbolu oraz stanu zwracający wynik w formie symbolu taśmy (7 bitów) oraz kierunku ruchu głowicy (1 bit);
		\item[\texttt{STATE TRANSLATOR}] -- Blok pamięci symulujący funkcję przejścia dla zadanego symbolu oraz stanu zwracający wynik w formie stanu (8 bitów);
		\item[\texttt{STATE REGISTER}] -- Rejestr przechowujący obecny stan maszyny;
		\item[\texttt{HEADER POSITION}] -- Zbiór liczników binarnych z możliwością inkrementacji oraz dekrementacji reprezentujący usytuowanie głowicy na taśmie (wskazuje obecnie wybrany symbol w \texttt{TAPE});
		\item[\texttt{HEADER POSITION DIRECTION DISASSEMBLY}] -- Prosty multiplekser sygnału zmiany wartości \texttt{HEADER POSITION} w zależności od kierunku zmiany;
		\item[\texttt{HEADER POSITION CLOCK SYNCHRONIZATION}] -- Bramka AND synchronizująca sygnał zmiany wartości \texttt{HEADER POSITION} z zegarem;
		\item[\texttt{STEP EXECUTION STATE REGISTER}] -- Rejestr przechowujący obecny stan wykonywania ruchu maszyny. Ze względu na brak praktycznego rozwiązania jednoczesnego odczytu i zapisu symbolu taśmy, przeprowadzenie przejścia zostało podzielone na dwa ruchy rozdzielające te operacje; stąd dwa stany wykonania ruchu;
		\item[\texttt{IN-PROGRAM CONTROL LINES BUFFERS}] -- Zestaw buforów oddzielający kontrolne linie sygnałowe generowane przez interfejs zewnętrzny od tych wewnęrznych;
		\item[\texttt{IN-OPERATION CONTROL LINES BUFFER}] -- Bufor oddzielający wewnętrzne linie kontrolne od tych generowanych przez urządzenie podczas pracy; %Rozdzielenie to jest potrzebne w celu umożliwienia programowania maszyny (dostępu sygnałów zewnętrznych do wewnętrznych);
	\end{description}
	
	\subsubsection{Sygnały}
	Oznaczenie sygnałów jest takie same jak oznaczenia zastosowane w schematach KiCad. Sygnały z $\overline{\texttt{poprzeczką nad nazwą}}$ oznaczają, że sygnał jest aktywny dla niskiego poziomu, zaś nieaktywny dla wysokiego.
	%Część sygnałów nie ma bezpośredniego zastosowania podczas pracy maszyny, jednak są one potrzebne do jej programowania i odczytu wyniku pracy. Te sygnały znajdują się w interfejsie zewnętrznym, który został opisany w następnym podrozdziale. 
	\newline \newline
	\textbf{Lista sygnałów}:
	\begin{description} 
		\item[\texttt{D\textsubscript{0...7}}] -- Magistrala danych używana do programowania i odczytu  danych z maszyny;
		\item[\texttt{S\textsubscript{0...6}}] -- Magistrala symbolu taśmy;
		\item[\texttt{SR\textsubscript{0...6}}] -- Magistrala symbolu taśmy zapisanego w \texttt{TAPE SYMBOL REGISTER};
		\item[\texttt{Q\textsubscript{0...7}}] -- Magistrala stanu;
		\item[\texttt{QR\textsubscript{0...7}}] -- Magistrala stanu zapisanego w \texttt{STATE REGISTER};
		\item[\texttt{H\textsubscript{0...15}}] -- Magistrala z wartością licznika \texttt{HEADER POSITION};
		\item[\texttt{CLK}] -- Sygnał zegarowy wyznaczający rytm pracy; każda zmiana sygnału ze stanu niskiego do wysokiego powoduje przejście \texttt{STEP EXECUTION STATE REGISTER} do następnego stanu; działa pod warunkiem aktywnego sygnału \texttt{CLK\_EN};
		\item[\texttt{CLK\_EN}] -- Kontroluje dopływ sygnału \texttt{CLK} do \texttt{STEP EXECUTION STATE REGISER};
		\item[\texttt{DIR}] -- Ustala kierunek zmiany wartości \texttt{HEADER POSITION}. Wysoki stan sygnału znaczy inkrementację, niski stan dekrementację;
		\item[\texttt{HP\_EN}] -- Dopuszcza możliwość zmiany wartości \texttt{HEADER POSITION};
		\item[\texttt{HP\_EN\_CLK}] -- sygnał \texttt{HP\_EN} zsynchronizowany z zegarem (\texttt{CLK}): $\texttt{HP\_EN\_CLK} = \texttt{HP\_EN} \cdot \texttt{CLK}$; synchronizuje zmianę wartości \texttt{HEADER POSITION} z zegarem;
		\item[\texttt{DIR\_UP}] -- Decyduje o inkrementacji wartości \texttt{HEADER POSITION}; działa przy przejściu ze stanu niskiego do wysokiego;
		\item[\texttt{DIR\_DOWN}] -- Decyduje o dekrementacji wartości \texttt{HEADER POSITION}; działa przy przejściu ze stanu niskiego do wysokiego;
		\item[$\overline{\texttt{RESET}}$] -- Resetuje urządzenie, czyli ustawia \texttt{STEP EXECUTION STATE REGISTER} do stanu $0$ oraz ustawia wartość \texttt{HEADER POSITION} na $0$;
		\item[$\overline{\texttt{PROGRAM}}$] -- Ustala tryb pracy. Poziom niski sygnału oznacza tryb programowania maszyny, poziom wysoki zaś tryb wykonywania programu;
		\item[\texttt{PROGRAM}] -- Przeciwieństwo sygnału $\overline{\texttt{PROGRAM}}$; sygnał ten jest równoważny $\overline{\texttt{OPERATE}}$ -- podany został jedynie dla czytelności schematu;
		\item[$\overline{\texttt{OPERATE}}$] -- Przeciwieństwo sygnału $\overline{\texttt{PROGRAM}}$. Niski poziom oznacza pracę urządzenia;
		\item[\texttt{DDIR}] -- Ustala kierunek przepływu danych z wewnętrznych modułów wybranych sygnałami \texttt{CS\textsubscript{0}}, \texttt{CS\textsubscript{1}} na magistralę \texttt{D\textsubscript{0...7}}. Wysoki poziom sygnału oznacza zapis danych z inteferjsu zewnętrznego do modułów wewnątrz urządzenia. Niski poziom oznacza odczyt;
		\item[\texttt{CS\textsubscript{0}}, \texttt{CS\textsubscript{1}}] -- Kombinacja tych sygnałów wybiera wewnętrzny moduł do odczytu/zapisu danych z zewnętrznego interferjsu (kierunek zależny od \texttt{DDIR});
		\item[$\overline{\texttt{DT}}$] -- Włącza przesył danych między wewnętrznymi modułami a intefrejsem zewnętrznym; aktywny dla niskiego poziomu;
		\item[\texttt{TSR\_WR}] -- Zapisuje dane z magistrali symbolu taśmy \texttt{S\textsubscript{0...6}} do \texttt{TAPE SYMBOL REGISTER}; zapis odbywa się podczas przejścia sygnału ze stanu niskiego do wysokiego;
		\item[\texttt{SR\_WR}] -- Zapisuje dane z magistrali stanu \texttt{Q\textsubscript{0...7}} do \texttt{STATE REGISER}; zapis odbywa się podczas przejścia sygnału ze stanu niskiego do wysokiego;
		\item[\texttt{T\_$\overline{\texttt{OE}}$}] -- Włącza wyjście danych z \texttt{TAPE} na magistralę \texttt{S\textsubscript{0...6}}; aktywny dla niskiego poziomu;
		\item[\texttt{T\_$\overline{\texttt{WE}}$}] -- Włącza zapis danych w \texttt{TAPE} z magistrali \texttt{S\textsubscript{0...6}}; aktywny dla niskiego poziomu;
		\item[\texttt{SYT\_$\overline{\texttt{OE}}$}] -- Włącza wyjście danych z \texttt{SYMBOL TRANSLATOR} na magistralę \texttt{S\textsubscript{0...6}} oraz sygnał \texttt{DIR\_B}; aktywny dla niskiego poziomu;
		\item[\texttt{SYT\_$\overline{\texttt{WE}}$}] -- Włącza zapis danych w \texttt{SYMBOL TRANSLATOR} z magistrali \texttt{S\textsubscript{0...6}} oraz sygnału \texttt{DIR\_B}; aktywny dla niskiego poziomu;
		\item[\texttt{STT\_$\overline{\texttt{OE}}$}] -- Włącza wyjście danych ze \texttt{STATE TRANSLATOR} na magistralę \texttt{Q\textsubscript{0...7}}; aktywny dla niskiego poziomu;
		\item[\texttt{STT\_$\overline{\texttt{WE}}$}] -- Włącza zapis danych w \texttt{STATE TRANSLATOR} z magistrali \texttt{Q\textsubscript{0...7}}; aktywny dla niskiego poziomu;
		\item[\texttt{QR\_$\overline{\texttt{OE}}$}] -- Włącza wyjście danych ze \texttt{STATE REGISTER} na magistralę \texttt{D\textsubscript{0...7}}; aktywny dla niskiego poziomu;
		\item[\texttt{T\_$\overline{\texttt{*E}}$\_E}, \texttt{SYT\_$\overline{\texttt{*E}}$\_E}, \texttt{ST\_$\overline{\texttt{*E}}$\_E}, \texttt{QR\_$\overline{\texttt{OE}}$\_E}] -- Sygnały odpowiadające tym bez sufiksu ,,\texttt{\_E}'', będące połączone z nimi przez bufor \texttt{U18} oraz będące wynikiem dekodera wyboru funkcji \texttt{U19};
		\item[\texttt{T\_$\overline{\texttt{*E}}$\_B}, \texttt{SYT\_$\overline{\texttt{OE}}$\_B}, \texttt{STT\_$\overline{\texttt{OE}}$\_B}, \texttt{DIR\_B}] -- Sygnały odpowiadające tym bez sufiksu ,,\texttt{\_B}'', będące połączone z nimi przez bufor \texttt{U17}. Są one sygnałami generowanymi przez maszynę podczas obliczeń potrzebnymi do poprawnej jej samodzielnej pracy;
		\item[$\overline{\texttt{SYMB\_BUF}}$] -- Włącza przepływ danych z \texttt{TAPE} lub \texttt{SYMBOL TRANSLATOR} (w zależności od wybranego urządzenia sygnałami \texttt{CS\textsubscript{0,1}}) na magistralę \texttt{D\textsubscript{0...7}}; aktywny dla niskiego poziomu;
		\item[$\overline{\texttt{STAT\_BUF}}$] -- Włącza przepływ danych ze \texttt{STATE TRANSLATOR} na magistralę \texttt{D\textsubscript{0...7}}; aktywny dla niskiego poziomu;
		\item[$\overline{\texttt{FINISH\_STATE}}$] -- Informuje o osiągnięciu przez maszynę stanu akceptującego i zarazem zatrzymującego pracę; aktywny dla niskiego poziomu.
		\item[\texttt{SEST\_STATE}] -- Wskazuje obecny stan \texttt{STEP EXECUTION STATE REGISTER};
	\end{description}
	
	\subsubsection{Interfejs zewnętrzny}
	W celu zasilenia urządzenia oraz komunikacji zastosowany został zewnętrzny interfejs składający się z 16 wyjść sterujących, 17 informujacych o stanie maszyny oraz 11 zasilających. W sumie daje to 44 wyjścia. Wszystkie wspomniane sygnały (po za \texttt{Vcc} oraz \texttt{Vss}) zostały wyjaśnione w poprzednim podrozdziale. Wyjścia \texttt{Vcc} oraz \texttt{Vss} są wyjściami odpowiednio zasilania i uziemienia. \texttt{Vcc} powinno mieć napięcie +5V względem \texttt{Vss}. \\
	By uniknąć zbyt podobnych oznaczeń, kierunki sygnałów zamiast ,,Wej/Wyj'' zostały oznaczone angielskimi ,,In/Out''.
	\newline \newline
	\textbf{Tabela sygnałów interfejsu:}
	\SetTblrInner{rowsep=4pt}
	\begin{table}[h!]
		\centering
		\begin{tblr}{||c|c|c||}
			\hline
			\textbf{Num. wyjścia} & \textbf{Sygnały} & \textbf{Kierunek} \\ \hline
			1-8 & \texttt{D\textsubscript{0...7}} & In/Out \\ \hline
			9 & \texttt{CLK} & In \\ \hline
			10 & $\overline{\texttt{RESET}}$ & In \\ \hline
			11 & $\overline{\texttt{PROGRAM}}$ & In \\ \hline
			12 & \texttt{DDIR} & In \\ \hline
			13-14 & \texttt{CS\textsubscript{0,1}} & In \\ \hline
			15 & $\overline{\texttt{DT}}$ & In \\ \hline
			16 & $\overline{\texttt{FINISH\_STATE}}$ & Out \\ \hline
			17,19,21,23,25 & \texttt{Vcc} & --- \\ \hline
			18,20,22,24,26 & \texttt{Vss} & --- \\ \hline
			27-42 & \texttt{H\textsubscript{0...15}} & Out \\ \hline
			43 & \texttt{SESR\_STATE} & Out \\ \hline
			44 & \texttt{Vss} & --- \\ \hline
		\end{tblr}
		\caption{Lista sygnałów interfejsu zewnętrznego}
		\label{tab:extinterfacetable}
	\end{table}
	\newpage
	
	\subsubsection{Programowanie}
	Mając na uwadze czytelność dokumentacji zdecydowano o opisie stanu sygnałów w czasie w formie wykresu czasowego. Każdy rząd odpowiada konkretnemu sygnałowi; linia w rzędzie zmieniająca się między stanem wysokim a niskim oznacza stan sygnału w czasie. Każda kolumna wykresu przedstawia konkretny odcinek czasowy wraz ze stanami sygnałów w tym czasie. Sygnały zmieniane są podczas przejść między odcinkami czasowymi. 
	\newline \newline
	Programowanie maszyny odbywa się za pomocą interfejsu zewnętrznego. Dzięki odpowiedniemu wykorzystaniu udostępnionych sygnałów użytkownik jest w stanie manipulować zawartością bloków pamięci \texttt{TAPE}, \texttt{SYMBOL TRANSLATOR} oraz \texttt{STATE TRANSLATOR}.
	\newline \newline
	\textbf{Przygotowanie maszyny do programowania} \\
	W celu rozpoczęcia programowania maszyny należy wpierw ją do tego przygotować. Zadanie to jest dość proste - wystarczy ustawić sygnał $\overline{\texttt{PROGRAM}}$ na 0, po czym puścić puls sygnału $\overline{\texttt{RESET}}$ w stanie również 0.
	\begin{figure}[h!]
		\begin{tikztimingtable}[
			timing/slope=0.1,
			timing/coldist=10pt,
			xscale=2,yscale=2,
			thick
			]
			$\overline{\texttt{PROGRAM}}$	& HLLLLLLLLLLLLLLLLL \\
			$\overline{\texttt{RESET}}$		& HHLHHHHHHHHHHHHHHH \\		
			\extracode
			\begin{pgfonlayer}{background}
				\begin{scope}[gray,semitransparent,semithick]
					\vertlines{1,...,17}
				\end{scope}
				\begin{scope}[gray,semitransparent,semithick]
					\horlines[dotted]{}
				\end{scope}
				\foreach \n in {1,2,...,\twidth}
					\draw (\n-0.5,-\nrows-.8)
						node [below,inner sep=2pt] {\scalebox{.75}{\tiny\n}};
			\end{pgfonlayer}
		\end{tikztimingtable}
		\caption{Schemat czasowy przygotowania maszyny do programowania}
		\label{fig:programprimer}
	\end{figure}
	
	\textbf{Programowanie modułu \texttt{TAPE} (taśmy)} \\
	Programowanie modułu \texttt{TAPE} zaczyna się przygotowaniem go do programowania, czyli wybraniem modułu, z użyciem sygnałów \texttt{CS\textsubscript{0,1}}, oraz wybraniem kierunku przepływu danych sygnałem \texttt{DDIR} (1 dla zapisu). \\
	Następnie zapis słowa wejściowego taśmy odbywa się w pętli: ustala się sybol do zapisu na magistrali \texttt{D\textsubscript{0...7}}, aktywuje się moduł pulsem sygnału  $\overline{\texttt{DT}}$ oraz wysła się dwa pulsy zegara \texttt{CLK} w celu przejścia do następnej komórki pamięci przechowującej zawartość taśmy. \\
	Na wykresie czasowym kolumny 1-2 reprezentują przygotowanie \texttt{TAPE} do programowania. 2-6 i 7-11 przedstawiają dwa pojedyncze zapisy symboli do dwóch kolejnych komórek pamięci w \texttt{TAPE}. Sygnały w kolumnach 13-17 zostały wykropkowane w celu oznaczenia wielokrotnego powielenia operacji zapisu symbolu.
	\begin{figure}[h!]
		\begin{tikztimingtable}[
			timing/slope=0.1,
			timing/coldist=10pt,
			xscale=2,yscale=2,
			thick
			]
			\texttt{CS\textsubscript{0}}	& ULLLLLLLLLLLLLLLLU \\
			\texttt{CS\textsubscript{1}}	& ULLLLLLLLLLLLLLLLU \\
			\texttt{DDIR}					& UHHHHHHHHHHHHHHHHU \\
			\texttt{D\textsubscript{0...7}}	& UU DDDDD{}DDDDD{};[dotted]DDDDD{}; U \\
			$\overline{\texttt{DT}}$		& HH HLHHH  HLHHH  ;[dotted]HLHHH; H \\
			\texttt{CLK}					& HH HHLHL  HHLHL  ;[dotted]HHLHL; H \\
			\extracode
			\begin{pgfonlayer}{background}
				\begin{scope}[gray,semitransparent,semithick]
					\vertlines{1,...,17}
				\end{scope}
				\begin{scope}[darkgray,semitransparent,thick]
					\vertlines{2,7,12,17}
				\end{scope}
				\begin{scope}[gray,semitransparent,semithick]
					\horlines[dotted]{}
				\end{scope}
				\foreach \n in {1,2,...,\twidth}
					\draw (\n-0.5,-\nrows-4.8)
						node [below,inner sep=2pt] {\scalebox{.75}{\tiny\n}};
			\end{pgfonlayer}
		\end{tikztimingtable}
		\caption{Schemat czasowy programowania \texttt{TAPE}}
		\label{fig:programtape}
	\end{figure}
	
	\textbf{Programowanie modułu \texttt{SYMBOL TRANSLATOR}}
	
	\section{Fizyczny Opis Działania}
	
\end{document}
